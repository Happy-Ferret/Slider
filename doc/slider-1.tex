\documentclass{latex2man}

\begin{Name}{1}{slider}{Jesse McClure}{pdf presenter}{Slider - PDF presentation tool}

\Prog{Slider} - PDF presentation tool

\end{Name}

\section{Synopsis}
slider
	\oOptArg{-F}{ config}
	\oOptArg{-c}{ class}
	\Arg{presentation}
	\oArg{notes}

\section{Description}

\Prog{Slider} is a pdf slideshow presentation tool that can be used on a
single monitor, or with an optional and configurable presenter mode for
multiple outputs (e.g. laptop screen and projector).

\section{Options}

\begin{description}
\item[\OptArg{-F}{ config}] Override the default configuration file
\item[\OptArg{-c}{ class}] Specify the resource class under which
resources will be obtained from the configuration.  This allows for a
single configuration file to contain settings for different
configurations.  An example use would be a single configuration file
which contains a default set of bindings, and bindings for a "Remote"
class which may select key bindings to work with a presention remote.
\item[\Arg{presentation}] The primary pdf file to be used for the
presentation (required).
\item[\Arg{notes}] Secondary pdf file of presenter notes that will be
shown on the local screen (for two outputs only).
\end{description}

\section{Configuration}

Configuration is implemented via an X resources data base file which is
read on startup.  A well-commented example configuration file is
distributed with \Prog{slider} and can be found at
\File{/usr/share/slider/config}.
Slider checks the following paths for user configuration and uses only
the first file found:
\File{$XDG_CONFIG_HOME/slider/config},
\File{$HOME/.config/slider/config},
\File{/usr/share/slider/config}.

\section{Commands}

Slider is controlled by the commands described below.  These commands
can be triggered by key or button bindings as set in the configuration
file, or the commands can be sent to the "Command" property of slider's
main window with a tool like xprop.  The commands can be abbreviated to
the first four characters (or three for dot).

Several of these commands allow or require additional parameters which
are provided as a white-space delineated list.  Several commands require
a color definition which is provided as four numeric values between 0.00
and 1.00 to define the red, green, and blue color components as well as
an alpha (opacity) value.

\begin{description}
\item[next] Move to the next slide.
\item[previous] Move to the previous slide.
\item[quit] End the slide show.
\item[redraw] Redraw the current slide.
\item[mute] Mutes the display.  And optional parameter can specify
"black" (default) or "white" muting, or provide a full RGBA color
specifier. (Not implemented yet).
\item[sorter] Display the "slide sorter" view.
\item[pen] Draw on the current slide with a selected color and line
width.  Requires five numeric parameters separated by whitespace.  The
first four define the color.  The fifth parameter defines the line width.
\item[dot] Set a 'dot' cursor for the mouse.  This can provide the same
effect as a laser pointer, but with much better control.  As 'pen' this
requires five numeric parameters.  The first four define the color, the
fifth defines a radius.
\item[custom] Set a custom cursor like 'dot' but rather than a circular
dot, specify a text string to use as a cursor.  This requires five
numeric and one string parameter.  The first four specify the color as
in 'dot' and 'pen', the fifth specifies a font size, and the sixth
provides the text string.
\item[action] Activate action/media links.  With no additional
parameters, this command will show a standard mouse cursor and wait for
a click event to select an action/media link.  A single integer
parameter can be provided to specify a link (the ordinal number of that
link on the current page).  A link type can also be specified to
activate the first link of the specified type (not yet implemented).
\item[fullscreen] Toggle the fullscreen state of the main presentation
window.
\item[zoom] Zoom in on a portion of the window.  If numeric color
parameters are specified, they will define how a bounding box is to be
drawn with the mouse - slider will then wait for a region to be selected
with the mouse for zooming into.  Zoom can also accept "quad N"
parameters upon which it will zoom into the specified quadrant of the
current page.
\end{description}

\section{Author}
Copyright \copyright 2012-2014 Jesse McClure \\
License GPLv3: GNU GPL version 3 \URL{http://gnu.org/licenses/gpl.html} \\
This is free software: you are free to change and redistribute it. \\
There is NO WARRANTY, to the extent permitted by law.

Submit bug reports via github: \\
\URL{http://github/com/TrilbyWhite/slider.git}

I would like your feedback.  If you enjoy \Prog{Slider} see the bottom
of the site below for detauls on submitting comments: \\
\URL{http://mccluresk9.com/software.html}

\LatexManEnd
